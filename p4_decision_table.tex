\documentclass[12pt]{article}
\usepackage[margin=1in]{geometry}
\usepackage{booktabs}
\usepackage{array}
\usepackage{titlesec}
\usepackage{enumitem}
\usepackage{helvet}
\usepackage{hyperref}
\renewcommand{\familydefault}{\sfdefault}

\titleformat{\section}{\normalfont\large\bfseries}{\thesection.}{0.5em}{}
\title{Decision Table for Library Book Lending Policy}
\author{Chandanpreet Singh 40324569\\
SOEN 6481: Software Systems Requirements Specification\\
Submitted via GitHub repository: \href{https://github.com/chandanpreet707/soen6481-d1-vision}{soen6481-d1-vision}
}
\date{\today}

\begin{document}
\maketitle

\section{Context}

As part of the requirements specification for a university Library Management System, we are designing a rule-based policy to decide whether a user can borrow a specific book. The system will take into account both the user type and the book type to determine if borrowing is allowed and for how long.

\section{Rules to Consider}

\begin{itemize}
    \item \textbf{User Types:} Student, Faculty
    \item \textbf{Book Types:} Regular, Reference
    \item Reference books cannot be borrowed by anyone.
    \item Borrow durations for regular books:
    \begin{itemize}
        \item Student: 14 days
        \item Faculty: 30 days
    \end{itemize}
\end{itemize}

\section{Conditions and Actions}

\textbf{Conditions:}
\begin{itemize}
    \item C1: User Type = Student or Faculty
    \item C2: Book Type = Regular or Reference
\end{itemize}

\textbf{Actions:}
\begin{itemize}
    \item A1: Allow Borrowing = Yes or No
    \item A2: Duration = 14, 30, or N/A
\end{itemize}

\section{Decision Table (Complete)}

\begin{center}
\begin{tabular}{>{\bfseries}l l l c c}
\toprule
Rule & User Type & Book Type & Allow Borrowing & Duration \\
\midrule
R1 & Student & Regular   & Yes & 14 days \\
R2 & Student & Reference & No  & N/A     \\
R3 & Faculty & Regular   & Yes & 30 days \\
R4 & Faculty & Reference & No  & N/A     \\
\bottomrule
\end{tabular}
\end{center}

\section{Analysis}

\subsection*{Redundancy Check}
The table includes all valid combinations. While R2 and R4 both result in ``No'' for Reference books, they are not redundant because they correspond to different user types and maintain clarity.

\subsection*{Conflicts}
There are no conflicting rules. Each user/book combination is handled clearly with consistent actions.

\subsection*{Simplification Possibility}
Since reference books are never allowed to be borrowed, we can extract that as a precondition:

\begin{quote}
\textit{If Book Type is Reference → Deny borrowing (regardless of user type)}
\end{quote}

This allows us to simplify the table and focus only on Regular books for rule evaluation.

\section{Simplified Decision Table (Regular Books Only)}

\begin{center}
\begin{tabular}{>{\bfseries}l l c c}
\toprule
Rule & User Type & Allow Borrowing & Duration \\
\midrule
R1 & Student & Yes & 14 days \\
R2 & Faculty & Yes & 30 days \\
\bottomrule
\end{tabular}
\end{center}

\section{Conclusion}

This decision table helps formalize the borrowing policy and can directly support implementation in the Library Management System. By identifying a precondition for Reference books, we reduce complexity and improve maintainability.

\end{document}
