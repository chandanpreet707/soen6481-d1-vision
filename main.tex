\documentclass[12pt]{article}
\usepackage{geometry}
\geometry{a4paper, margin=1in}
\usepackage{titlesec}
\titleformat{\section}{\normalfont\Large\bfseries}{\thesection}{1em}{}
\titleformat{\subsection}{\normalfont\large\bfseries}{\thesubsection}{1em}{}
\usepackage{graphicx}
\usepackage{hyperref}
\hypersetup{
    hidelinks
}
\usepackage{longtable}

\title{Vision Document\\\large CanaLease (CA)}
\author{Chandanpreet Singh \\ Student ID: 40324569 \\ SOEN 6481 -- Systems Requirement Specification}
\date{July 11 2025 }

\begin{document}

\maketitle
\tableofcontents
\newpage

\section{Introduction}
The purpose of this document is to present the vision for CanaLease (CA), a web-based property management platform designed for the Canadian rental market. This document outlines the problem space, key stakeholders, proposed solution, and the motivation behind developing the platform.

CanaLease (CA) aims to simplify and digitize the daily operations of landlords, tenants, finance staff, and system administrators involved in managing residential or commercial rental properties. The platform addresses challenges such as rent collection, lease management, maintenance tracking, financial reporting, and communication between all involved parties.

This document follows the Rational Unified Process (RUP) guidelines and serves as an initial point of reference for the development team, stakeholders, and project sponsors. It will be updated throughout the project lifecycle as more details are gathered as the system evolves.


\section{Problem Domain}

\subsection{Background and Context}
The housing and rental markets rely on traditional methods for background checks, manual paperwork, delayed emails, and brief spreadsheets. These outdated methods lead to inefficiency, data inconsistencies, and poor property manager and tenant experience. Due to the lack of a centralized digital system, it is sometimes difficult to track maintenance requests, manage payments, and produce accurate financial reports for taxation.   

\subsection{Problem Statement}
The current condition of property management is affected by the following:
\begin{itemize}
    \item Inefficient methods for collecting and tracking rental agreements.
    \item Miscommunication between the landlord and the tenants.
    \item Manual maintenance request system.
    \item Ineffective financial reporting and documentation.
    \item Absence of role-based access control for different user types.

\end{itemize}
These problems lead to increased operational expenses, postponed services, tenant dissatisfaction, and revenue loss. An integrated system is required to simplify the process of a property management system.

\subsection{Stakeholders}
The stakeholders impacted by these problems:

\begin{itemize}
    \item \textbf{Landlords/Property Managers}: Need for streamlined tools to manage tenants, track leases, automate rent collection, and oversee maintenance.
    \item \textbf{Tenants}: Need for transparency, timely maintenance responses, and easy payment options.
    \item \textbf{Accounting and Finance Personnel}: Access to accurate financial data, automated reporting, and tax compliance support.
    \item \textbf{System Administrators}: Manage user roles, access rights, and ensure secure configuration of the platform.
\end{itemize}

\section{Solution Domain}

\subsection{Product Vision}
CanaLease(CA) will be a web-based platform, accessible from any device, built by keeping in mind the requirements of landlords, tenants, finance staff, and administrators. The goal for building this web app is to reduce manual work, improve communication, and make day-to-day property management tasks faster and more reliable. The system will work and be protected by a secure web interface and scalable database for future upgrades. It will support both small landlords and larger companies in managing a few units to multiple buildings.  

\subsection{Key Features}
The system will include the following core features:

\begin{itemize}
    \item \textbf{Online Rent Payments:} Tenants can pay rent securely online; landlords can track payments and send reminders for late payments.
    \item \textbf{Lease and Tenant Management:} Landlords can create, upload, and manage lease agreements, and track lease start/end dates.
    \item \textbf{Maintenance Request System:} Tenants can submit maintenance issues by filling online forms, attaching photos; landlords/managers can assign workers and track status easily.
    \item \textbf{Financial Reports:} Generates monthly and yearly reports for income, expenses, and tax filing.
    \item \textbf{Role-Based Access Control:} Each user (tenant, landlord, accountant, admin) will have access only to what they need.
    \item \textbf{Notifications and Reminders:} Automatic emails or SMS for rent due, lease renewals, or pending maintenance.
    \item \textbf{Multi-language Support:} English and French language web interface to support Canadian users.
\end{itemize}

\subsection{Benefits and Impact}
This web-application (CanaLease) will help in reducing dependency on paperwork, minimizing response times, providing users transparency in various property management operations. Landlords will spend less time in managing documents, and tracking rentals. Tenants will have smoother experience with reliable support. Finance teams will have access to organized records for financial reporting. Overall, the whole system will improve trust, interaction, and satisfaction between landlords and tenants, also in saving time and avoiding errors.  

\section{Stakeholders and Users}
CanaLease (CA) is built by keeping in mind the requirements of a wide range of users, providing an interactive, easy-to-use, and adaptive user interface (UI) while performing complex operations seamlessly. Below is a breakdown of the main user types and their needs.

\subsection{Landlords and Property Managers}
They will be responsible for managing rental units, leases, tenants, and maintenance. Their main goals include:
\begin{itemize}
    \item Keeping track of tenant information, their unit, and during of stay.
    \item Collecting rent payments, also charging/collecting late fees
    \item Resolving maintenance requests
    \item Access to reports about income and expenses
\end{itemize}
CanaLease will help them save time, reduce paperwork, and stay on top of property operations.

\subsection{Tenants}
Tenants use the system to pay rent, request repairs, and view lease information. Their needs include:
\begin{itemize}
    \item Paying rent easily and securely online through various payment methods
    \item Reporting maintenance problems and track status online at their convenience
    \item Getting reminders for upcoming rent or lease renewals
    \item Viewing lease agreements or past payments
\end{itemize}
The goal is to give tenants a better experience and faster support from landlords/property managers.

\subsection{Accounting and Finance Personnel}
These users handle bookkeeping, tax preparation, and financial reporting. They need:
\begin{itemize}
    \item Access to organized financial data
    \item Tools to generate reports for taxes, income, and expenses
    \item Integration with payment systems and bank records
\end{itemize}
CanaLease will reduce errors and make it easier to meet financial and legal requirements.

\subsection{System Administrators}
Admins are responsible for setting up accounts and user access, and permissions. Their needs are:
\begin{itemize}
    \item Creating and configuring user accounts
    \item Assigning roles and permissions
    \item Ensuring system security and access control
\end{itemize}
CanaLease will offer an admin dashboard to handle user roles efficiently and securely.


\section{Creativity and Critical Thinking}

CanaLease (CA) is designed with Canadian users in mind. It handles all basic property managements tasks, while providing additional features like: 

\subsection{Support for Bilingual Users}
Since Canada has two official languages, the system will support both English and French. Users can switch between language through interactive User Interface(UI) based on their preference.

\subsection{Seasonal Maintenance Reminders}
Properties in Canada often deal with seasonal issues like snow removal, heating system checks, or frozen pipes. CanaLease will include built-in reminders and templates for seasonal maintenance to help landlords stay ahead of weather-related problems.

\subsection{Eviction and Legal Notice Tools}
In some provinces, legal steps must be followed for evictions or rent increases. CanaLease can include province-specific templates and reminders to help landlords follow local regulations properly.

\subsection{Scalability for Multi-Unit and Multi-Property Management}
While small landlords may manage a few units, larger companies could manage dozens of buildings. CanaLease is designed to scale — allowing users to switch between properties, group units by building, and manage different portfolios under one account.

\subsection{Possible Future Enhancements}
\begin{itemize}
    \item Integration with government services for tax reporting
    \item Mobile app for quicker access
    \item AI-based rent price recommendations based on location and market trends
    \item Chat feature between tenant and landlord within the app
\end{itemize}

These features aren’t just extra — they show how CanaLease can grow into a full platform that fits Canadian laws, user habits, and future expectations.

\newpage
\appendix
\section*{Appendix A: Time Log}

\begin{longtable}{|p{0.3\textwidth}|p{0.15\textwidth}|p{0.2\textwidth}|p{0.3\textwidth}|}
\hline
\textbf{Section} & \textbf{Time Spent (hrs)} & \textbf{Date} & \textbf{Notes} \\
\hline
Introduction & 1.5 & 2025-07-11 & Wrote Purpose, Scope, and Definitions \\
Problem Domain & 1.5 & 2025-07-11 & Wrote problem background and simplified stakeholder descriptions \\
Solution Domain & 1.5 & 2025-07-11 & Wrote product vision, features, and benefits section \\
Stakeholders and Users & 1.2 & 2025-07-11 & Wrote user roles and needs for CanaLease (CA) \\
Creativity and Critical Thinking & 1.2 & 2025-07-11 & Wrote custom ideas for CanaLease including bilingual support, weather-based features, and scalability \\
\hline
\end{longtable}


\end{document}
